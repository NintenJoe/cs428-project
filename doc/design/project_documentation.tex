%	@file FinalDocumentation.tex
%
%	@author Josh Halstead, Joe Ciurej
%	@date Spring 2014
%
%   LaTeX source file for the design document for the CS428 project.  More
%   information about the requirements for this documentation file can be found
%   here: "https://wiki.engr.illinois.edu/display/cs428sp14/Final+Documentation"

\documentclass{article}
\usepackage[parfill]{parskip}		% Package that improves paragraph formatting
\usepackage[pdftex]{graphicx}		% Package that facilitates figure insertion
\usepackage{float}					% Package that allows for arbitrary figure insertion
\usepackage{hyperref}				% Package that allows for text hyperlinking
\usepackage{tikz}					% Package that facilitates box rendering
\usepackage{color}

% The following code block changes a few parameters of the page to make the
% text fill up more space on any given page.  For our liking, the default 'article'
% template is too sparse.
\addtolength{\voffset}{-2cm}
\addtolength{\textheight}{4cm}
\addtolength{\oddsidemargin}{-1.5cm}
\addtolength{\textwidth}{2cm}

% This command creates blue boxes around class names.
% @see "http://tex.stackexchange.com/questions/36401/drawing-boxes-around-words"
\newcommand{\classname}[1] {\texttt{#1}}
\newcommand{\methodname}[1] {\texttt{#1}}
\newcommand{\projectname}[0] {\texttt{Zol} }

\newcommand{\hreffsm}[1] {\href{http://en.wikipedia.org/wiki/Finite-state\_machine}{{\color{blue}\underline{#1}}}}
\newcommand{\hreftemplate}[1] {\href{http://en.wikipedia.org/wiki/Template\_method\_pattern}{{\color{blue}\underline{#1}}}}
\newcommand{\hrefmvc}[1] {\href{http://en.wikipedia.org/wiki/Model\%E2\%80\%93view\%E2\%80\%93controller}{{\color{blue}\underline{#1}}}}
        \newcommand{\hrefloz}[0] {\href{http://en.wikipedia.org/wiki/The\_Legend\_of\_Zelda\_(video\_game)}{{\color{blue}\underline{The Legend of Zelda}}}}
        \newcommand{\hrefxp}[0] {\href{http://www.extremeprogramming.org/}{{\color{blue}\underline{Extreme Programming}}} }

\newcommand{\insertdiagram}[2]
{
	\begin{figure}[H]
		\centering
		\fbox{\includegraphics[height=#2]{figures/#1}}
		\caption{UML Diagram for the \classname{#1} Structure}
	\end{figure}
}

\begin{document}

	% Title Page %
	\title{Zol: Design Document}
	\author{J. Halstead, J. Ciurej, A. Exo, N. Jeffrey, E. Christianson, E. Chan \\
		University of Illinois: Urbana-Champaign (CS428)}
	\date{\today}
	\maketitle

	% Table of Contents %
	\tableofcontents
	\clearpage

	% Body %
	\section[Overview]{Brief Overview of \projectname}
	The primary motivation behind the initial conception and the continued
	development of the \projectname project was a desire among the members
	of the original development team to create a general and extensible game
	engine that could be used to facilitate rapid game development and game
	prototyping.  That said, the objective of the \projectname project is
	to provide game designers and developers with an intuitive and robust game
	engine groundwork upon which they can quickly and easily develop video
	games with a wide variety of different rules and behaviors.

	While the project implementation has quite a way to go before it can be
	used to easily generate completely general games\footnote{See the
	``Future Work'' section for more details}, the project in its current state
	supports a great assortment of tools for creating varied two-dimensional
	experiences.  In particular, \projectname provides the following utilities
	for two-dimensional games:

	\begin{itemize}
		\item General game entity construction and specification using the
            \hreffsm{finite state machine} behavior specification pattern.

		\item General and overridable collision detection and collision
		resolution infrastructures.

		\item Composite hitbox support with abstracted SVG specification and
		arbitrary collision resolution behavior per hitbox.

		\item Animation support with specification via the commonly used
		\href{http://en.wikipedia.org/wiki/Sprite\_(computer\_graphics)#Sprites\_by\_CSS}
        {{\color{blue}\underline{sprite sheet technique}}}.

		\item Tile-based game world construction and generation using adjustable
		input image maps.

		\item Integrated support for in-game cameras with panning and easing
		functionality.

		\item Basic infrastructure for generalized game world rendering with
		support for user interface widgets and overlays.
	\end{itemize}

	In order to demonstrate these capabilities, we've included the implementation
	of a basic game that mimics the classic title \hrefloz.  While this version
	isn't nearly as fully featured as the original, we believe that this demo game
	adequately demonstrates both the capabilities of our engine and how to properly
	utilize these capabilities.


	\section[Development]{\projectname Development Process}
	The development process followed by the development team while developing
	\projectname was a variant of the \hrefxp process (as it was presented in
	CS427: Software Engineering I) adapted for use in an academic setting.
	The adaptations (which are expounded upon in more explicit detail below)
	were made to better suit the development process to a group developers with
    highly variate schedules and (consequently) limited availability to meet.

		\subsection{Iterative Development}
		As far as iterative development is concerned, our process doesn't deviate
		much from its base Extreme Programming model.  Each development iteration
		spans two weeks, beginning with a planning game (involving the resolution
		and refinement of user stories) and ending with an iteration product.

		In order to stress Extreme Programming's core principle of communication,
		our process also requires that each iteration contain four group meetings:
		two group requirement/review meetings (one at the beginning of each week
		to iron out requirements and design), one preparation meeting (the day
		before the end of the iteration to polish the iteration product), and
		one presentation meeting (the final day of the iteration to demonstrate
		the product).

		\subsection{Refactoring}
		The process we used for the \projectname project is identical to Extreme
		Programming process with respect to its approach to refactoring, requiring
		that development follow a test-code-refactor cycle and that code be
		refactored whenever and wherever necessary.  This requirement led the
		development cycle of each team member during each iteration to adhere
		to a schedule similar to the following:

		\begin{itemize}
			\item Refactor the implementation code associated with one's currently
			assigned task, updating test methods wherever necessary.

			\item Write the implementation and associated testing code to
			accomplish one's currently assigned task.

			\item Perform first-pass refactorings to the previously written code
			to improve its quality (time-permitting).
		\end{itemize}

		\subsection{Testing}
		While our process for \projectname values and upholds the core Extreme
		Programming rules related to software testing, it includes one slight
		modification: \href{http://en.wikipedia.org/wiki/Test-driven\_development}
        {{\color{blue}\underline{test-driven development}}} practices are replaced with
		\href{http://treyhunner.com/2013/07/test-inspired-development/}
        {{\color{blue}\underline{test-inspired development}}} practices.  Put another way, our process
		removes the requirement imposed by Extreme Programming for tests to be
		written before implementation code and instead allows developers to
		choose on a case-by-case basis on which segment of code should be
		written first.

		The primary reason that our process was adapted in this way was to
		reduce the volatility of testing code.  Many of us found that, unlike
		advocates of the test-first methodology claim, writing test code first
		doesn't adequately elucidate implementation design requirements, which
		often causes tests to be rewritten or overhauled after such requirements
		are discovered (often by means of writing the implementation code itself).
		As such, we adopted the more flexible approach of test-inspired
		development over test-first development to eliminate the amount of time
		wasted on this unnecessary rehashing, which expedites project development
		in most cases.

		\subsection{Collaborative Development}
		Perhaps the most prominent deviation of our process from its Extreme
		Programming base is its replacement of the centralized code review
		technique called \href{http://www.extremeprogramming.org/rules/pair.html}
        {{\color{blue}\underline{pair-programming}}} with the distributed code review scheme afforded
		by \href{https://www.atlassian.com/git/workflows#!pull-request}
        {{\color{blue}\underline{pull requests}}} and the \href{https://help.github.com/articles/using-pull-requests#shared-repository-model}
        {{\color{blue}\underline{shared-repository model}}}.  This change shifts collaborative
		development to become asynchronous and spatially independent, supporting
		code reviews in a much more flexible and accessible way.

		While retaining the pair programming requirement in our process would
		have certainly been possible, it would have introduced a lot of
		scheduling overhead.  Since the pair programming technique assumes that
		the members of each development pair have similar schedules, it's far
		too cumbersome to be useful in an academic project involving multiple,
		diverse student members (e.g. the \projectname project).  Hosting the
		reviews through a distributed version control system (i.e. GitHub)
		eliminates this need for scheduling while retaining all the benefits
		of code reviews, providing the optimal collaboration management solution.


	\section[Requirements]{\projectname Project Requirements}
	The requirements for the \projectname project can be found on
    \href{https://github.com/NintenJoe/zol/wiki}{{\color{blue}\underline{the project's main wiki page}}}.
	It's important to note that these requirements are directed towards a game
	designer client as opposed to a game player client.  As such, these
	requirements detail the required capabilities of a general game engine that
	a designer may use to create a game (and not any particular game created
	through this means).  In contrast, the given use case document outlines
	the potential user actions for the given \projectname example game.  This
	choice was made in order to cut down on the amount of abstraction and to
	present a more cohesive use case to potential designer users.

	The following is a listing of important requirement specification documents
	found on the main project wiki:

	\begin{itemize}
        \item \href{https://github.com/NintenJoe/zol/wiki#project-requirements-user-stories}{{\color{blue}\underline{User Stories}}}
        \item \href{https://wiki.engr.illinois.edu/download/attachments/233410967/use-cases.pdf}{{\color{blue}\underline{Use Cases}}}
	\end{itemize}


	\section[Architecture and Design]{Implementation Design and Architecture}

	\insertdiagram{Zol}{3.0in}

	At a high level, the \projectname game engine is an implementation of the
    \hrefmvc{model-view-controller} design pattern with the following components:

	\begin{description}
		\item[Model:] All the logic and data associated with the implementation
		of the underlying game world, including the logic for updating entities,
		loading new worlds, detecting entity collisions, resolving entity
		collisions, etc.
		\item[View:] All the logic and data associated with the rendering and displaying
		of the contents of the underlying game world, including the logic for
		displaying world entities, displaying world tiles, and rendering user
		interface modules and overlays.
		\item[Controller:] All the logic and data associated with updating the
		model and view components based on user input.  In contrast to the
		previous two components, this component is quite sparse, encapsulating
		only the user key input capture and broadcast logic.
	\end{description}

	The most interesting and complex of these components is the model component,
	which contains all the logic for game behaviors.  The main class within this
	component is the \classname{GameWorld} type, which represents a singular and
	whole instance of a simulated world in which a game may take place.  Each
	of these world instances consists of four primary sub-components: a list of
	\classname{Entity} objects that live and interact within the world (each of
	which is represented as a \hreffsm{finite automaton}), a \classname{World}
	object to serve as the tile-based backdrop for the space, a \classname{Camera}
	to keep track of the observer's view into the world, and a
	\classname{CollisionDetector} to detect and resolve collisions that occur
	between entities.  These components work in tandem to simulate interactions
	between \classname{Entity} instances within the world, which serves as the
	basis for game behavior.

		\subsection[Primary Components]{Primary Implementation Components}
		The core of the \projectname project consists of the
		higher-level types used to implement the model and view components within
		the overall architecture.  For each of these primary components, the listing
		below describes the major functionality encapsulated by that primary component
		and provides a visual aid for its role in the \projectname architecture through a
        \href{http://en.wikipedia.org/wiki/Unified\_Modeling\_Language}{{\color{blue}\underline{UML diagram}}}.

			\subsubsection[\classname{Event}]{The \classname{Event} Component}
			The \classname{Event} class is a representation of an event that occurs
			within the context of a \classname{GameWorld}.  An \classname{Event}
			instance is generated whenever an event of interest occurs (e.g. a
			collision, a user input, etc.) with an appropriate type (as specified
			by its \classname{EventType}) and propagated through the world to the
			relevant (i.e. affected) \classname{Entity} objects.  Instances of this
			type drive the state changes and (consequently) behaviors of the
			\classname{Entity} objects to which they're propagated.

			\insertdiagram{Event}{2.0in}

			The primary function of the \classname{Event} class is to serve as
			a form of message that encapsulates information about an unusual
			occurrence within the simulation that's likely to cause change.
			Instances of this type are passed back and forth between the
			\classname{GameWorld} and its contained \classname{Entity} objects
			to facilitate communication without coupling the two types.

			\subsubsection[\classname{PhysicalState}]{The \classname{PhysicalState} Component}
			The \classname{PhysicalState} class represents the physical components
			of any object within the context of the \classname{GameWorld}, such as
			their position, velocity, mass, and collision volume (or hitbox).
			The properties for any \classname{PhysicalState} can be modified by
			applying state deltas, which are encoded instances of the
			\classname{PhysicalState} containing delta values for each property.
			This functionality of the type is often exercised when incremental
			changes are being applied to a persistent instance (which is often
			the case for \classname{Entity} updates).

			\insertdiagram{PhysicalState}{3.0in}

			The \classname{PhysicalState} type is used primarily by the
			\classname{Entity} class to represent the physical properties of
			each individual \classname{Entity} instance.  This type is also
			utilized by the \classname{StateMachine} class in order to
			indicate physical changes generated by the behaviors of contained
			\classname{State} instances.

			\subsubsection[\classname{StateMachine}]{The \classname{StateMachine} Component}
			The \classname{StateMachine} class serves as an implementation of
			the \hreffsm{finite state machine} behavior specification pattern for
			the \projectname project.
			In this particular implementation of the pattern, each \classname{State}
			represents a particular behavior codified by a change in
			\classname{PhysicalState} (e.g. standing still, moving, etc.)
			and each \classname{Transition} represents a change of state invoked
			by a particular \classname{Event} or class of \classname{Event} instances.

			\insertdiagram{StateMachine}{2.0in}

			The main purpose of the \classname{StateMachine} type is to structure
			and specify the behavioral patterns of \classname{Entity} class
			instances in a completely general way.  Essentially, the
			\classname{StateMachine} type serves as the `brains' for the
			\classname{Entity} instances in the world, dictating how these
			objects should behave and react under certain circumstances.

			\subsubsection[\classname{Entity}]{The \classname{Entity} Component}
			The \classname{Entity} class represents any dynamic object that
			exists within the game world and interacts with other similar
			objects.  Each instance of this type contains two distinct
			sub-components: a \classname{PhysicalState} (which represents the
			object's current physical properties) and a \classname{StateMachine}
			(which represents the object's current behavioral properties).
			The \classname{StateMachine} is primarily responsible for governing
			how the \classname{Entity} behaves and changes as it interacts with
			other objects within the same \classname{GameWorld} and responds to
			events.

			\insertdiagram{Entity}{2.0in}

			All \classname{Entity} objects have static asset binding support
			by default based on their classification, which is used to specify
			an instance's \classname{StateMachine} infrastructure and animation/hitbox
			bindings to inner \classname{State} instances.  This feature can
			be used to easily generate new \classname{Entity} types on the fly
			with unique behaviors and visual elements.  The following list
			describes the paths and formats for these static bindings:

			\begin{itemize}
				\item \textbf{State Machine:} \$(ZOL\_DIR)/assets/data/entities/(entity-class).json
				\item \textbf{State Animation:} \$(ZOL\_DIR)/assets/graphics/entities/(entity-class)/(entity-state).png
				\item \textbf{State Hitbox:} \$(ZOL\_DIR)/assets/data/hitbox/(entity-class)/(entity-state).svg
			\end{itemize}

			\subsubsection[\classname{World}]{The \classname{World} Component}
			The \classname{World} class is responsible for loading all of the levels (we call the segments). 
			A level in our game engine is a collection of segments. Segments are areas that the 
			player can explore and they contain obstacles and enemies. Players can transition between 
			segments by walking over transition tiles. To make segment creation easy we save them as 
			.gif files. Each pixel represents a tile, and we also define which tiles are tangible 
			(the player collides with them). Additionally we define where entities such as the player 
			and enemies spawn. The \classname{Segment} class loads these image files and provides and interface 
			for retrieving this information. The \classname{Level} class is a container for the segments. It matches 
			up the transition tiles between segments in the same level. The \classname{World} class searches the 
			directory where we save segments, and creates levels out of them.

			\insertdiagram{World}{2.0in}

			TODO

			\subsubsection[\classname{GameWorld}]{The \classname{GameWorld} Component}
			The \classname{GameWorld} type is the core type within the model
			component of the overall project architecture, encapsulating all
			the functionality associated with simulating game logic.  Each
			\classname{GameWorld} is comprised of a collection of distinct
			\classname{Entity} instances, which represent the dynamic contents
			of the environment, and a single \classname{World}, which represents
			the static, tile-based backdrop for the environment.  Each \classname{GameWorld} instance
			is responsible for detecting and resolving the interactions between
			its contained objects, which involves generating and propagating
			\classname{Event} instances when extraordinary events occur (e.g.
			collisions, user inputs, etc.), intercepting and interpreting
			\classname{Event} instances generated by contained \classname{Entity}
			objects, and detecting/resolving collisions between \classname{Entity}
			objects and the background environment.

			\insertdiagram{GameWorld}{3.0in}

			The primary purpose of the \classname{GameWorld} type is to amalgamate
			all the model components of the \projectname architecture into a
			single, cohesive type.  This type is used by the controller and view
			components of the architecture to capture an overview of the current
			state of the game model at any point in time and to modify this state
			as needed.

			\subsubsection[\classname{GameView}]{The \classname{GameView} Component}
			The \classname{GameView} type serves as the the core type of the
			view component of the overall project architecture, containing all
			the logic associated with rendering model information and user
			interface elements.  Given an instance of the \classname{GameWorld}
			type, the \classname{GameView} type is capable of rendering all
			the \classname{Entity} and \classname{World} contents of this
			instance relative to the world's current viewport.  This type
			is also capable of rendering arbitrary model-dependent user
			interface overlays to provide the player with more precise
			information about the game's current state.

			\insertdiagram{GameView}{2.0in}

			Much like the \classname{GameWorld} type for the architecture's
			model components, the primary purpose of the \classname{GameView}
			type is to combine all the view components of the \projectname architecture
			into a single, cohesive type.  This type is used solely by the controller
			to render the contents of the \classname{GameWorld} after these
			contents have been updated each frame.

		\subsection{Framework/Library Integration}
		For the implementation of our project, we consciously avoided using any
		existing game development frameworks in order to maintain complete
		implementation freedom.  One of the primary goals of our project was
		to implement a generalized game engine, and using an existing game
		framework would only serve to complicate and limit this task.  Instead,
		we decided to implement the entirety of our system without a framework,
		only using supplementary libraries such as
        \href{http://www.pygame.org/news.html}{{\color{blue}\underline{pygame}}} to help with lower-level
		operations (e.g. image loading, image rendering, etc.).  Thus, the
		\projectname project's design is completely independent of any existing
		game framework's design/architecture.


	\section[Future Plans]{The Future of \projectname}
	During our initial development period, we found that we were not able to
	fully realize many of the features we wished to incorporate into the
	\projectname game engine.  The following section details the major additions
	we wish to integrate into the engine in the future as well as future plans
	we have for the project.

		\subsection[Features]{Future Functions and Features}
		There were many features that we considered during the initial development
		period that couldn't be fully implemented due to time constraints.  These
		features primarily involve improving the generality of the existing game
		engine, extending the basic functions provided by the game engine (e.g.
		adding to the list of provided state behaviors), and providing an enhanced
		user interface for specifying world entity behavior and game world
		infrastructure.

		The following is a listing of all the most important features to be added
		to the \projectname project in the future:

		\begin{itemize}
			\item Creating a user interface to facilitate the full manipulation
			of entity state machines.

			\item Creating a user interface to facilitate more user-friendly
			creation and manipulation of tile-based game worlds.

			\item Further generalization of the collision detection and collision
			resolution systems.

			\item Further generalization of the rendering engine to incorporate
			arbitrary user interface overlays and widgets.

			\item Adding support for arbitrary game state saving and loading.

			\item Adding support for hot-swap game loading to facilitate more
			rapid prototyping and game parameter adjustment.

			\item Expanding the functionality of the in-game camera utility to
			allow for behaviors such as zooming.

			\item Expanding of the in-game menu system to support more general
			menus with a wider variety of options.

			\item Restructuring the existing architecture to facilitate
			multi-threading (e.g. separate threads for the model-related and
			the view-related logic).
		\end{itemize}

		\subsection[Deployment]{Future Deployment Details}
		At the end of the initial development period, we plan to publish
		the \projectname project as a stand-alone application to a new
		GitHub project repository.  Once the project has been republished,
		the development process for the project will shift to a
		\href{http://en.wikipedia.org/wiki/The\_Cathedral\_and\_the\_Bazaar}
        {{\color{blue}\underline{Bazaar-based process}}} with Joe as the primary project manager
		and a new phase of development will begin.  During this phase and all
		future stages of development, the project repository will be made publicly
		available so that any existing team member or open-source contributor
		may help to expand the existing implementation into a more robust
		game engine.

		\subsection[Personal Reflections]{Team Reflections on the Project}
		The following section contains a brief overview of each team member's
		personal experience with the project, including their personal reflections
		on both the project's development (especially in terms of the development
		process) and on the final product.

		\begin{description}
			\item[Josh's Reflection] \hfill \\
			When Joe and I wrote the proposal for Zol we knew we wanted to
			use a development process that was similar to XP but offered more
			flexibility in terms of TDD and pair programming. Our compromise
			was to use Git, adopt code reviews via GitHub's pull requests,
			and adhere to ``test-inspired development''.

			What (pleasantly) surprised me about those small adaptations of
			XP was how much more connected to the entire team and project I
			felt. The code review process especially was transparent and
			open to the entire team. It offered me the chance to see and
			comment on changes that would have been masked by Subversion and
			pair programming. This alone left me with a far greater
			understanding and feeling of ownership of the project as a whole.
			And I would gladly use a similar process in the future.

			As for the final product, I am quite pleased with the architecture
			of the engine. The quality of which is demonstrated by how
			quickly we can create demos and alter the game environment. What
			I learned in creating the final demo is that even the most well
			thought out systems need modifications once they are actively
			being used. That lesson is something I will take with me as
			I work on future products and struggle with getting it just right
			the first time vs.\ just quickly getting a product and refactoring
			from there.

			\item[Andrew's Reflection] \hfill \\
			I enjoyed working on this project, and I was surprised by how effectively 
			we were able to work together as a team. The elements of our process that 
			I found to be the most beneficial were the pull-request/code review 
			contribution method and coveralls. Github's pull-request feature made it 
			really simple to review the work that each member did. We could give feedback 
			to each other, and it also provides a form of documentation of our process. 
			Coveralls provided us with code coverage information. This made it easy to 
			see which areas of our code our tests were missing. Even though we allowed 
			tests to be written after code, we ended up with 95 percent code coverage.

			Now that we're done, I'm satisfied with what we created. It's not exactly a 
			game, but it is a robust game engine. Creating new levels, animations, or 
			even character behaviour is incredibly easy. All of these things can be 
			added without touching the code. 

			Overall I feel like I have learned a lot about being part of a development 
			team. I tend to work alone even when given the option to work with others, 
			so this was a very valuable experience for me. Now I feel much more comfortable 
			contributing to a shared repository using a version control system. In the future 
			I will use this experience to more effectively and more confidently contribute to 
			open source projects.

			\item[Nick's Reflection] \hfill \\
			I think that the development process we used for this project was
			really swell. We made excellent use of branches in Git to work through
			our assigned tasks without breaking things for other people working on
			their tasks, but while still being able to push things to Gitub for
			version control. We also had Travis for continuous integration, which
			helped avoid pushing broken code to master, and coveralls to help us
			know which classes needed better test coverage. Coveralls is by no means
			perfect though, since it only tracks whether lines were touched during
			testing, not whether they behave as intended, but it served as a very
			good heuristic that kept everyone writing good tests for their code.
			These factors contributed to a final product that I think is very stable
			and well tested.

			\item[Edwin's Reflection] \hfill \\
			I've been very passionate about video games for a long time and this
			is the first time I've taken part in the creating side of it with a group of
			others. Overall I thoroughly enjoyed the entire experience and learned a lot
			about software development process. I also learned a lot about the the fun
			and pain of game development. The development process was very good
			and orderly in that I learned a lot about using a shared repository and all the
			goods it can bring as well as the headaches. The final product is something
			I am really proud of, if we had more time I wish we could have added
			sound effects but other than that, it is something I am satisfied with.

			\item[Eric's Reflection] \hfill \\
			I like the development process that we used worked out pretty well,
			and it was a very good decision to switch to Git for version control. I
			am pretty satisfied with how our end result came out, because its
			a very well structured game engine that can be repurposed for multiple
			games. It took more time to have a game to show for it, but it was a
			good decision overall.

			\item[Joe's Reflection] \hfill \\
			Overall, I'd say that I'm quite pleased with the outcome of this
			project.  While there were certainly a few high points and low points
			during development, I'm happy with the experience overall because it
			taught me a lot about collaborative software development and game
			development technologies.

			In particular, I learned that the shared repository model is extremely
			effective in facilitating distributed collaborative development and
			that test-inspired development serves as a great testing technique
			alternative to test-driven development when dealing with unfamiliar
			designs and technologies.  Additionally, I thought that the adapted
			development process that our group used for this project worked really
			well, and I hope that I can utilize similar processes effectively
			in my future projects.
	\end{description}

\end{document}

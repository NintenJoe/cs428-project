%	@file FinalDocumentation.tex
%
%	@author Josh Halstead, Joe Ciurej
%	@date Spring 2014
%	
%   LaTeX source file for the design document for the CS428 project.  More 
%   information about the requirements for this documentation file can be found
%   here: "https://wiki.engr.illinois.edu/display/cs428sp14/Final+Documentation"

\documentclass{article}
\usepackage[parfill]{parskip}		% Package that improves paragraph formatting
\usepackage[pdftex]{graphicx}		% Package that faciliates figure insertion
\usepackage{float}					% Package that allows for arbitrary figure insertion
\usepackage{hyperref}				% Package that allows for text hyperlinking
\usepackage{tikz}					% Package that facilitates box rendering

% The following code block changes a few parameters of the page to make the
% text fill up more space on any given page.  For our liking, the default 'article'
% template is too sparse.
\addtolength{\voffset}{-2cm}
\addtolength{\textheight}{4cm}
\addtolength{\oddsidemargin}{-1.5cm}
\addtolength{\textwidth}{2cm}

% This command creates blue boxes around class names.
% @see "http://tex.stackexchange.com/questions/36401/drawing-boxes-around-words"
\newcommand{\classname}[1] {\texttt{#1} }
\newcommand{\methodname}[1] {\texttt{#1} }
\newcommand{\projectname}[0] {\texttt{Zol} }

\newcommand{\hreffsm}[1] {\href{http://en.wikipedia.org/wiki/Finite-state\_ machine}{#1}}
\newcommand{\hreftemplate}[1] {\href{http://en.wikipedia.org/wiki/Template\_ method\_ pattern}{#1}}
\newcommand{\hrefmvc}[1] {\href{http://en.wikipedia.org/wiki/Model\%E2\%80\%93view\%E2\%80\%93controller}{#1}}
\newcommand{\hrefloz}[0] {\href{http://en.wikipedia.org/wiki/The\_ Legend\_ of\_ Zelda\_ (video\_ game)}{The Legend of Zelda}}

\newcommand{\insertdiagram}[2]
{
	\begin{figure}[H]
		\centering
		\fbox{\includegraphics[height=#2]{figures/#1}}
		\caption{UML Diagram for the #1 Structure}
	\end{figure}
}

\begin{document}

	% Title Page %
	\title{Zol: Design Document}
	\author{Joshua Halstead and Joseph Ciurej \\
		University of Illinois: Urbana-Champaign (CS428)}
	\date{\today}
	\maketitle

	% Table of Contents %
	\tableofcontents
	\clearpage

	% Body %
	\section[Overview]{Brief Overview of \projectname}
	The primary motivation behind the initial conception and the continued
	development of the \projectname project was a desire among the members
	of the original development team to create a general and extensible game 
	engine that could be used to facilitate rapid game development and game
	prototyping.  That said, the objective of the \projectname project is 
	to provide game designers and developers with an intuitive and robust game 
	engine groundwork upon which they can quickly and easily develop video
	games with a wide variety of different rules and behaviors.

	While the project implementation has quite a way to go before it can be
	used to easily generate completely general games\footnote{See the 
	``Future Work'' section for more details}, the project in its current state 
	supports a great assortment of tools for creating varied two-dimensional 
	experiences.  In particular, \projectname provides the following utilities 
	for two-dimensional games:

	\begin{itemize}
		\item General game entity construction and specification using the 
		\hreffsm{finite state machine} behavior specification pattern.

		\item General and overridable collision detection and collision
		resolution infrastructures.

		\item Composite hitbox support with abstracted SVG specification and 
		arbitrary collision resolution behavior per hitbox.

		\item Animation support with specification via the commonly used
		\href{http://en.wikipedia.org/wiki/Sprite\_ (computer\_ graphics)#Sprites\_ by\_ CSS}
		{sprite sheet technique}.

		\item Tile-based game world construction and generation using adjustable 
		input image maps.

		\item Integrated support for in-game cameras with panning and easing
		functionality.

		\item Basic infrastructure for generalized game world rendering with 
		support for user interface widgets and overlays.
	\end{itemize}

	In order to demonstrate these capabilities, we've included the implementation
	of a basic game that mimics the classic title \hrefloz.  While this version
	isn't nearly as fully featured as the original, we believe that this demo game
	adequately demonstrates both the capabilities of our engine and how to properly
	utilize these capabilities.

	\section[Development]{\projectname Development Process}
	TODO

		\subsection{Iterative Development}
		TODO

		\subsection{Refactoring}
		TODO

		\subsection{Testing}
		TODO

		\subsection{Collaborative Development}
		TODO

	\section[Requirements]{\projectname Project Requirements}
	TODO

	\section[Architecture and Design]{Implementation Design and Architecture}
	TODO

		\subsection[Primary Components]{Primary Implementation Components}
		TODO

			\subsubsection{TODO}
			TODO

			\subsubsection{TODO}
			TODO

		\subsection{Framework/Library Integration}
		TODO

	\section[Future Plans]{The Future of \projectname}
	\label{sec:future}
	TODO

		\subsection[Features]{Future Functions and Features}
		TODO

		\subsection[Deployment]{Future Deployment Details}
		TODO

		\subsection[Personal Reflections]{Team Reflections on the Project}
		TODO

		\begin{description}
			\item[Josh's Reflection] \hfill \\
			TODO

			\item[Andrew's Reflection] \hfill \\
			TODO

			\item[Nick's Reflection] \hfill \\
			TODO

			\item[Edwin's Reflection] \hfill \\
			TODO

			\item[Eric's Reflection] \hfill \\
			TODO

			\item[Joe's Reflection] \hfill \\
			TODO
		\end{description}

\end{document}
